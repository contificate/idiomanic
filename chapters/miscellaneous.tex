
\chapter{Miscellaneous}

	\paragraph{Brief Overview}

	This chapter is mostly for me to write about things I believe constitute
	good programming style and practice in C++.  

	\section{Restraining objects to a valid state}
		
		If you've ever used classes from the STL, you'll know that many of them
		implement \textit{states} (in the form of state flags, most notably
		\texttt{std::ios\_base::iostate}) as their way of informing the
		programmer of their state in order to allow for error-checking logic to
		be implemented around their usage. This style of programming is often a
		lot more useful than adopting the cultural norms of, say, a language
		like Java; where it's often the case that error-checking logic only
		exists in the form of catching an exception.
		
		The state of many of these STL classes can be accessed using explicit
		\textit{accessors} (essentially, typically, \texttt{const}-correct
		``\textit{getters}'') in the form of member functions like:
		\texttt{good()}, \texttt{fail()}, \texttt{bad()}, etc. However, these
		accessors are also conveniently implemented in the form of operator
		overloading, which allows for expressive - error-checking - code to be
		written.

		When paired with C++'s ability to allow for variable declarations within
		\textit{if} statement conditions, such operator overloads can be useful
		when invoked from implicit conversion (\textit{contextually
		convertible}). Let's look at an example:

		% restraining by state example
		\begin{minted}[linenos, frame=single, tabsize=2]{cpp}
if (auto ifs = std::ifstream("file.bin", std::ios::binary)) {
	// use ifs
}
		\end{minted}

		This code does more for us than probably meets the eye. The code will
		construct the instance of \texttt{std::ifstream} for us and then invoke
		\texttt{explicit operator bool() const;} on it to yield a boolean for
		the \texttt{if}-statement's condition. We also restrain \texttt{ifs} to
		its valid state (unless we, of course, corrupt it ourselves). The
		\texttt{ifs} instance is bound to the \textit{true} block of our
		conditional structure - meaning RAII will destruct it at the end of that
		block. This is a safer way to write code because we cannot access
		\texttt{ifs} outwith a block that's only executed if its state is
		\textit{valid} (i.e. its \texttt{bool} overload yields \texttt{true}).
		

	\section{Don't use \texttt{std::endl}}

		A lot of people new to C++ often find themselves writing code like:

		% std::endl example
		\begin{minted}[linenos, frame=single]{cpp}
std::cout << "Hello, world" << std::endl;
		\end{minted}

		without a real understanding of what \texttt{std::endl} actually does.
		Despite its misleading name, \texttt{std::endl} doesn't only just
		\textit{end the line}, it also flushes the stream (using
		\texttt{std::flush}). Such a flush in most trivial text-outputting
		scenarios is not desired behaviour and so it's recommended that you just
		use a newline (\texttt{\textbackslash n}) character instead (as shown
		below).

		\begin{minted}[linenos, frame=single]{cpp}
std::cout << "Hello, world\n";
		\end{minted}

		Also, when nesting, keep in mind that \textbackslash n is a single
		character. Thus, when passing it to a stream, one should wrap it
		properly within single apostrophes (e.g. instead of \texttt{<<
		"\textbackslash n";}, one should write \texttt{<< `\textbackslash n';}).
		This simply looks more natural in terms of code style (despite the fact
		that compilers will almost definitely optimise out the idea of
		designating a string literal relocation for a single character in such
		scenarios). 
