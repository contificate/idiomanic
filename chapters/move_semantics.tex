
\chapter{Move Semantics}

\textit{Move Semantics} refers to the \textit{moving} of - typically
resource-owning - reference types in C++. At its essence, movement refers to the
shallow copying and nullification of resource members from a class. 

\paragraph{Brief History}
In C++, it's quite often the case that we construct local objects and then
immediately pass and then assign them elsewhere. Before C++11, this would
prove to be an expensive way to write code due to overhead incurred during these
operations. Namely, the construction of a \textit{temporary} object that only
exists to be copied elsewhere leads to redundant overhead.

\section{Understanding Ownership}
	\paragraph{What is ownership?}

\section{\texttt{rvalue} reference syntax}

	C++11 introduced rvalue reference syntax, denoted by a type followed by two
	ampersands: \texttt{Type\&\&}. To understand why this syntax had to be
	introduced, one must recall fundamental rules governing a design decision
	made in the language a long time ago. This decision, briefly, was the
	decision to allow rvalues to bind to \texttt{const} references. 

	The concept of an rvalue is rather simple (ignoring more naunced value
	categories). The name \textit{rvalue} is derived from early terminology used
	by compiler developers to distinguish between each side of an assignment
	expression. A simple example of this can be seen below:

	% assignment example
	\begin{minted}[linenos, frame=single]{cpp}
int value = 1337;
	\end{minted}

	In this example, \texttt{value} is the \textit{lvalue} (left-hand side) and
	\texttt{1337} is the \texttt{rvalue} (right-hand side). The most high-level
	distinguishing feature of these value categories is simple: you can get the
	address of an lvalue, you can't get the address of an rvalue.\par

	We could get the address of \texttt{value} by dereferencing it, e.g.
	\texttt{\&value}. However, we cannot get the address of ``\texttt{1337}'' as
	it's an integer literal expression. We can consider the temporary nature of
	rvalues as temporary until \textit{bound}; where \textit{bound} refers to a
	temporary whose value has been bound to an identifier. We use rvalue
	reference syntax to extend the lifetime of temporaries by binding them to a
	type that essentially preserves the \textit{rvalueness}. Later on, we'll
	talk about specific cases of \textit{forwarding}, where we can retain the
	\textit{rvalueness} of parameterised values.
	
	\subsection{Binding rvalues to \texttt{const} references}	
	 
	 	As mentioned earlier, in C++ we have the ability to bind rvalues to
		\texttt{const} references. The rationale for this was likely something
		relating to safety within immutability. However, we won't linger on the
		motivations, we'll merely see how this language design decision has
		impacted the development of C++ since C++11.

		This rule is fundamental to truly understanding the reasoning behind the
		introduction of rvalue reference syntax. Thus, we'll begin with trivial
		examples and hopefully build up to less contrived examples.

		At its simplest, the rule can be summarised by the following code
		snippets:

		% 
		\begin{minted}[linenos, frame=single]{cpp}
void func(int& x);
// ...
int a = 45;
func(a); // valid, binding to an lvalue, a

func(34); // invalid, attempting to bind rvalue to non-const
          // reference
		\end{minted}

		As you can see, we cannot pass an rvalue to a function accepting a
		non-\texttt{const} reference parameter. However, let's see what happens
		if we change our function so that it accepts a \texttt{const} reference
		parameter:

		% 
		\begin{minted}[linenos, frame=single]{cpp}
void func(const int& x);
// ...
int a = 45;
func(a); // valid, binding to an lvalue, a

func(34); // valid, binding rvalue to const reference
		\end{minted}

		As you can see, both calls to \texttt{func} are allowed because we can
		bind both lvalues \textit{and} rvalues to \texttt{const} references. The
		above example may seem rather contrived but we can apply it to other
		scenarios where a \texttt{const} reference parameter is seen accepting
		rvalues. To do this, we must recall that rvalues can be yielded from
		other places. For example, \textit{conversion construction} yields an
		rvalue: 

		Let's define a trivial class that we can construct via conversion
		construction:
	
		% definition of conversion-constructible class	
		\begin{minted}[linenos, frame=single, tabsize=2]{cpp}
class Object {
public:
	Object(int x) : x_{x} { }

private:
	int x_;
};
		\end{minted}

		Since our constructor for \texttt{Object} isn't declared
		\texttt{explicit}, we can implicitly rely on conversion construction to
		construct an instance of \texttt{Object} when an \texttt{int} is
		supplied as an argument to a function accepting a \texttt{const}
		reference to \texttt{Object}:

		% passing our conversion-constructible class by const reference
		\begin{minted}[linesnos, frame=single, tabsize=2]{cpp}
void func(const Object& x);

func(45); // valid, conversion construction from int yields
          // an rvalue that can bind to the const reference
		\end{minted}

		A less contrived example of relying on \textit{conversion construction}
		for binding to \texttt{const} references can be seen in the rather
		common example of passing \texttt{std::string} by \texttt{const}
		reference.

		% passing std::string by const reference example	
		\begin{minted}[linenos, frame=single, tabsize=2]{cpp}
void func(const std::string& x);

func("foo"); // valid 
		\end{minted}

		In the above example using \texttt{std::string}, the string literal
		\texttt{"foo"} (i.e. \texttt{const char*}) is conversion constructed
		using a constructor provided by the class that \texttt{std::string} is a
		type definition (or alias declaration) of. E.g. \texttt{std::string} is
		a specialisation of \texttt{std::basic\_string<charT>} where
		\texttt{charT} = \texttt{char}. The constructor that is resolved in the
		above example is simply \texttt{std::string(const char*)} i.e.
		\texttt{std::basic\_string<charT = char>(const charT*)}.

		It's very important to note that we can also bind rvalues to parameters
		that are \textit{by-value}. This can be problematic when we attempt to
		introduce overloads for both \texttt{by-value} and by \texttt{const}
		reference parameters - as we'll see below:

		\begin{minted}[linenos, frame=single, tabsize=2]{cpp}
void func(int x);
void func(const int& x);

func(34); // pass in an rvalue
		\end{minted}

		When compiled with GCC, we get:

\begin{minted}[linenos, frame=single]{text}
example.cc: In function ‘int main()’:
example.cc:6:9: error: call of overloaded ‘func(int)’ is ambiguous
  func(34);
         ^
example.cc:2:6: note: candidate: void func(int)
 void func(int x);
      ^~~~
example.cc:3:6: note: candidate: void func(const int&)
 void func(const int& x);
      ^~~~
\end{minted}

		As you can see, the compiler considers our call to \texttt{func} to be
		\textit{ambiguos}. This is because both versions of \texttt{func} can
		accept rvalues. Ambiguity is the term assigned to a scenario where
		\textit{overload resolution} fails to choose between overloaded
		candidates.

		Given these overload resolution constraints, what must language
		designers do to work around ambiguity when they're intent on designating
		a type specifically for rvalues? The answer: they retain
		backwards-compatibility by introducing a new syntax and assigning it a
		higher \textit{precedence} for overload resolution to use as a
		resolution factor.

	\subsection{The Introduction of \texttt{rvalue} reference syntax}
		
		As noted at the end of the previous section, rvalue references were
		introduced largely to have a higher precedence for overload resolution
		when dealing with rvalues (as both by-value and by \texttt{const}
		reference parameters can already bind to rvalues; which leads to
		ambiguity when overloaded together).

		You can see the effect of that in the following snippet:

		\begin{minted}[linenos, frame=single, tabsize=2]{cpp}
// accept const reference
void func(const int& x) {
	std::cout << "const ref " << x << '\n';
}

// accept rvalue reference
void func(int&& x) {
	std::cout << "rvalue ref " << x << '\n';
}

// invoke with lvalue
int a = 45;
func(a);

// invoke will rvalue
func(37);
		\end{minted}

		The output of this code is simply:

		\begin{minted}[linenos, frame=single]{text}
const ref 45
rvalue ref 37
		\end{minted}

		We can see that, when provided with an lvalue, \texttt{a}, overload
		resolution opts for the \texttt{const} reference overload of
		\texttt{func}. When given an rvalue, the integer literal \texttt{37},
		the overload of \texttt{func} that accepts an rvalue reference is
		selected. 
		

	\subsection{Casting to \texttt{rvalues}}

	\subsection{Universal references}
		\subsubsection{Perfect forwarding}
